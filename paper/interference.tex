\documentclass[12pt]{article}

\usepackage[utf8]{inputenc}
\usepackage{xcolor}
\usepackage[plainruled,noline]{algorithm2e}
\usepackage{graphicx}
\usepackage{subfig}

\definecolor{myred}{rgb}{1, 0, 0}
\definecolor{myblue}{rgb}{0.22, 0.42, 0.63}

% --------- Todos and remarks
\newcommand\todo[1]{\textcolor{blue}{TODO: #1}}

% --------- typography
\newcommand\func[1]{\textsl{#1}}

% --------- miscellaneous
\newcommand\AMi{AMi}

% --------- algorithm2e
\SetKwProg{Fn}{def}{\string:}{}
\SetKwFunction{FComputeAMiInterferenceEdges}{\textnormal{\func{ComputeAMiInterferenceEdges}}}
\SetKwFunction{FPersistentInstructions}{\textnormal{\func{PersistentInstructions}}}
\SetKwFunction{FInstructions}{\textnormal{\func{Instructions}}}
\SetKwFunction{FBasicBlocks}{\textnormal{\func{BasicBlocks}}}
\SetKwFunction{FBasicBlock}{\textnormal{\func{BasicBlock}}}
\SetKwFunction{FConnected}{\textnormal{\func{Connected}}}
\SetKwFunction{FIsUse}{\textnormal{\func{IsUse}}}
\SetKwFunction{FUsers}{\textnormal{\func{Users}}}
\SetKwFunction{FUses}{\textnormal{\func{Uses}}}
\SetKwFunction{FDefs}{\textnormal{\func{Defs}}}
\SetKwFunction{FEdge}{\textnormal{\func{Edge}}}
\SetKwFunction{FIndex}{\textnormal{\func{Index}}}
\SetKwFunction{FOperands}{\textnormal{\func{Operands}}}
\SetKwFunction{FDef}{\textnormal{\func{Def}}}
\SetKw{Assert}{assert}
\SetKwFor{For}{for}{}{}
\SetKwSwitch{Switch}{Case}{Other}{switch}{}{case}{otherwise}{}{}
\SetKwIF{If}{ElseIf}{Else}{if}{}{else if}{else}{}%
\SetKwIF{If}{ElseIf}{Else}{if}{}{else if}{else}{}%
\SetKwInput{KwData}{Input}
\SetKwComment{Comment}{\# }{}


\begin{document}

\section{Problem Statement}

How to augment the interference graph to preserve program semantics for AMi-linearized programs?

\pagebreak

\section{Running Example}

\begin{figure}[h!]
  \centering
  \includegraphics[width=0.5\textwidth]{examples/cfg}
  \caption{\texttt{def-list} \brackets{\leftarrow\ \texttt{use-list}}}
\end{figure}

\begin{figure}[h!]
  \centering
  \subfloat[CASE 1\\BB0 < \textcolor{myred}{BB1} < \textcolor{myblue}{BB2} < BB3]{\includegraphics[width=0.35\textwidth]{examples/lin1}} \hfill
  \subfloat[CASE 2\\BB0 < \textcolor{myblue}{BB2} < \textcolor{myred}{BB1} < BB3]{\includegraphics[width=0.35\textwidth]{examples/lin2}}
  \caption{Linearization options}
  \label{fig:linearization-options}
\end{figure}

\section{Algorithm}

\subsection{Terminology}

\paragraph{Linearization}

Linearizing a control-flow graph creates a total order among basic blocks and their instructions.
Note that linearization differs from control-flow linearization.
The former defines the arrangement of basic blocks in memory and establishes a total order, while the latter removes secret-dependent branches but does not necessarily create a total order.

\paragraph{Join point}

In a control-flow graph, a join point is a location where multiple control paths in the program converge.
There can be multiple join points between two disconnected basic blocks.

\paragraph{Secret-dependent region}

% A secret-dependent region: (see also Libra paper for definition)
%     - A tuple of (set(basic block), set(dir-edge), entry, exit)
%     - Fully connected
%     - exit block postdominates all blocks
%     - terminator of entry is secret dependent

\subsection{Helper Functions}

\paragraph{Index(v)}

Functions to retrieve the index of an instruction and a basic block within that order defined by the applied lineariation.
% Note that index(i) can be compared with Index(bb) (Abuse of notation?)

\paragraph{Users(R, i)}

Returns all users (instructions) of the register defined by instruction i within region R.

\paragraph{Operands(v)}

Returns all operands of an instruction or all operands of the instruction in a basic block.

\paragraph{IsUse(op)}

Returns if the operand is a use

\paragraph{IsDef(op)}

Returns if the operand is a def

\paragraph{Def(op)}

Returns the defining instruction of the given operand.
For SSA form this is a single instruction.
For non-SSA form this can be multiple instructions.

\paragraph{LiveAt(v, bb)}

Return true if register v is live at bb

\paragraph{Instructions(R)}

Returns the instructions of the region.
A region can be a single basic block.

\paragraph{BasicBlock(i)}

Returns the basic block of the instruction i

\paragraph{Connected(bb1, bb2)}

Returns true if path from bb1 to bb2 or if path from bb2 to bb1.
(i.e., bb2 is reachable from bb1 or bb1 is reachable from bb2)

\paragraph{Edge(u, v)}

Represents a undirected edge (in the interference graph) between u and v.

\subsection{Algorithm}

The algorithm is applied after phi-elimination, meaning that the SSA form is not used.

\setlength{\algomargin}{0em}

\begin{algorithm}
\Fn{\FComputeAMiInterferences{$R$}}{
  \DontPrintSemicolon
  %\SetCommentSty{\textcolor{blue}}
  \KwData{Secret-dependent region $R$}
  \KwResult{Set of \AMi-induced interference edges $E$}
  $E \gets \emptyset$\;
  \For{$p \in \FInstructions{R}$ \mid \FIsPersistent(p)}{
    \For{$bb \in \FBasicBlocks{R} \mid \neg \FConnected(\FBasicBlock(p), bb)$}{

      \Comment{Case 1}
      \If{$\FIndex(bb) > \FIndex(\FBasicBlock(p))$}{

        \Comment{Case 1.1}
        \For{$user \in \FUsers(R, p)$}{
          \Assert{$\FIndex(user) < \FIndex(bb)\ \lor \FDefines(bb, p)$}\;
          \Comment{Else bug in \AMi{} tranformation?}
        }

        \Comment{Case 1.2}
        \For{$op \in \FOperands(bb) \mid \FIsUse(op)$}{
          \If{$\FIndex(\FDef(op)) < \FIndex(p)$}{
            $E \gets E \cup \{\FEdge(\FDef(op), p)\}$\;
            \Comment{LLVM: Add Segment(Range(p)) to LIS(Def(op))}
          } 
        }
      }

      \Comment{Case 2}
      \If{$\FIndex(bb) < \FIndex(\FBasicBlock(p))$}{

        \Comment{Case 2.1}
        \For{$op \in \FOperands(bb) \mid \FIsDef(op)$}{
          \For{$user \in \FUsers(\FDef(op)) \mid \FIndex(user) > \FIndex(p)$}{
            \Assert{$\FLiveAt(op, \FBasicBlock(p))$}\;
            \Comment{Else uninitialized var?}
          }
        }

        \Comment{Case 2.2}
        \For{$op \in \FOperands(p) \mid \FIsUse(op)$}{
          \If{$\FIndex(\FDef(op)) < \FIndex(bb)$} {
            \For{$i \in \FInstructions{bb}$} {
              $E \gets E \cup \{\FEdge(\FDef{op},i)\}$\;
              \Comment{LLVM: Add segment(bb) to LIS(p)}
            }
          }
        }
      }
    }
  }
  \Return{$E$}
}
\end{algorithm}

\pagebreak

\section{Issues}

\begin{itemize}
  \item
    Does the algorithm generalize to arbitrary control-flow graphs?
    Add more tests for each of the four categories to test edge cases.
  \item 
    Add support for spilling and splitting.
\end{itemize}

\section{Examples}

\subsection{Base Cases}

\begin{figure}[h!]
  \centering
  \subfloat[Case 1.2]{\includegraphics[width=0.35\textwidth]{examples/case_1_2}} \hfill
  \subfloat[Case 2.2]{\includegraphics[width=0.35\textwidth]{examples/case_2_2}}
  \caption{Cases X.2}
\end{figure}

\begin{figure}[h!]
  \centering
  \subfloat[Case 1.1]{\includegraphics[width=0.35\textwidth]{examples/case_1_1}} \hfill
  \subfloat[Case 2.1]{\includegraphics[width=0.35\textwidth]{examples/case_2_1}}
  \caption{Cases X.1}
\end{figure}

\pagebreak

\subsection{Variants}

\begin{figure}[h!]
  \centering
  \includegraphics[width=0.5\textwidth]{examples/case_2_1_1}
  \caption{Case 2.1.1}
\end{figure}

\end{document}
